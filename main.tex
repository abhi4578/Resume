\documentclass[a4paper,11pt]{article}
\usepackage{lmodern}
%-----------------------------------------------------------
\usepackage[top=0.5in, bottom=0.5in, left=0.5in, right=0.5in]{geometry}
\usepackage{graphicx}
\usepackage{url}
\usepackage{palatino}
\usepackage{tabularx}
\usepackage{hyperref}
\fontfamily{SansSerif}
\selectfont

\usepackage[T1]{fontenc}
\usepackage
%[ansinew]
[utf8]
{inputenc}

\usepackage{color}
\definecolor{mygrey}{gray}{0.75}
\textheight=9.75in
\raggedbottom

\setlength{\tabcolsep}{0in}
\newcommand{\isep}{-2 pt}
\newcommand{\lsep}{-0.5cm}
\newcommand{\psep}{-0.6cm}
\renewcommand{\labelitemii}{$\circ$}

\pagestyle{empty}
%-----------------------------------------------------------
%Custom commands
\newcommand{\resitem}[1]{\item #1 \vspace{-2pt}}
\newcommand{\resheading}[1]{{\small \colorbox{mygrey}{\begin{minipage}{0.975\textwidth}{\textbf{#1 \vphantom{p\^{E}}}}\end{minipage}}}}
\newcommand{\ressubheading}[3]{
\begin{tabular*}{6.62in}{l @{\extracolsep{\fill}} r}
	\textsc{{\textbf{#1}}} & \textsc{\textit{[#2]}} \\
\end{tabular*}\vspace{-8pt}}
%-----------------------------------------------------------

\begin{document}
\hspace{0.3cm}\\[-0.2cm]

%== HEADER ==%
\begin{center}      
    {\fontsize{16}{16}\selectfont Abhilash Venkatesh} \\ 
     (Alias Name: Abhilash V) \\
     Bengaluru, Karnataka, India \quad 
    \href {https://github.com/abhi4578}{Github}| \href {https://www.linkedin.com/in/abhi5782-/}{Linkedin} | \href {https://scholar.google.com/citations?user=8K_E_9EAAAAJ&hl=en}{Google Scholar} |
    \href{https://abhi4578.github.io}{Website}
    % You can also put your LinkedIn or website address
   % {\color{icnclr}\faLinkedinIn} \href{https://www.linkedin.com/}{linkedin.com/in/username}
\end{center}

\subsection*{Education}

\textbf{\href{https://omscs.gatech.edu/}{Georgia Institute of Technology}}\\
\textit{ Online /  Jan 2025 - Present} \\
Masters in Computer Science  \\
\newline
\textbf{\href{https://nitk.ac.in}{National Institute of Technology Karnataka, Surathkal}}\\
\textit{Mangalore, Karnataka, India / July 2016 - June 2020} \\
Bachelor of Technology in Information Technology  \\
\textbf{CGPA 9.08/10.0} 


\subsection*{Technical Leadership Work Experience} 
\textbf{Lead Engineer, Foundation for Science, Innovation and Development (FSID),\href{https://dataforpublicgood.org.in/}{Centre of Data for Public Good (CDPG)} (Formerly IUDX Programme Unit)} \\
\textit{Indian Institute of Science (IISc) campus, Bengaluru, Karnataka / April 2023 - January 2025} 
\begin{itemize}
\item Led Data Privacy research systems team in integrating SMPC based framework \href{https://carbynestack.io/}{Carbyne Stack} offline phase into SGX based TEE to accelerate offline phase 
\item Trained, delegated, and guided the DevOps team of five members on end-to-end Docker and Kubernetes (K8s) based  Data Exchange deployment.

\item Supervised the team of 3-5 members to the execution of two significant platform deployment migrations - updating the K8s platform and dataset ID migration with careful planning and testing resulting in at most half-day downtime.
\item Co-developed screening tests, assignment,and hired three candidates out of more than twenty candidates for DevOps role, 
\end{itemize}
\subsection*{Technical Work Experience}
\textbf{Senior Software Engineer, SID,  \href{https://iudx.org.in}{India Urban Data Exchange (IUDX) Programme Unit}} \\
\textit{IISc campus, Bengaluru, Karnataka / April 2022- March 2023} 
\begin{itemize}
\item Designed and implemented cloud cobhst optimization in AWS cloud, reducing the cost by 40\%.
\item  Proactively secured the platform during CVEs, especially Log4j2-shell attack, introduced secure coding practices through OpenSSF badge and operational security through strong passwords, two-factor \& hardened K8s clusters.
\end{itemize}
\textbf{Software Engineer, SID,  IUDX Programme Unit} \\
\textit{IISc campus, Bengaluru, Karnataka / July 2020- March 2022} 
\begin{itemize} 
\item Orchestrated Docker Swarm deployment of IUDX platform and infrastructure automation using Ansible.  
\item Researched and implemented clustering of databases and backend API servers using Kubernetes(K8s) in AWS and scaled to handle 1000 concurrent users.
\item Led a team of two members to deploy and support the end-to-end platform for external partners in the partner's own Azure K8s as a Service (AKS) environment. 
\end{itemize}


\subsection*{Research Experience}
\textbf{Part-Time Systems Engineer, SID,  R\&D IUDX Programme Unit \& FSID, CDPG} \\
\textit{Under \href{https://anshootandon.github.io/profile/}{Dr. Anshoo Tandon}, IISc campus, Bengaluru, Karnataka / March 2022 - March 2024}

\begin{itemize}
\item The YoLov5 Object Detection application is integrated into the system to run in SGX and AMD-based Trusted Execution environments (TEEs) with remote attestation flow and the Data Exchange system.
\item Designed, implemented and supported Secure Multi-Party Computing (SMPC) to run each party on different cloud providers, AWS and Azure, using docker containers. The experimental setup enabled \href{https://ieeexplore.ieee.org/document/10427509/}{memory optimization and scalability of neural network inference.}
\end{itemize}
\textbf {Summer Research Internship and Major Project} \\
\textit{Under \href{https://www.hesge.ch/hepia/annuaire/florent-gluck}{Prof. Florent Gluck}, HEPIA, Geneva; \href{https://infotech.nitk.ac.in/faculty/jaidhar-c-d}{Associate Prof. Jaidhar C.D}, NITK / May 2019 - June 2020}
\begin{itemize} 
     \item Worked on project titled \href{https://gitlab.com/remote-os-Image-clone-deployment}{Remote Operating System Image Clone Management (ROSIM)}. We used PXE technology, bash scripting ,  GRUB and a tiny Linux kernel built using Buildroot to remotely install/image OS to desktop computers over LAN. 
   \item Designed and implemented optimized OS imaging through caching, multicasting, and existing OS imaging.
	\item Developed Web-based management interface using Python-based Django framework for ease of OS image management.
\end{itemize}
\subsection*{Projects}
\textbf{\href{https://nitkit-vgst727-nppsa.nitk.ac.in/}{Protein Sequence Analyzer - Web Application}}  \\
  \textit{Under  \href{https://infotech.nitk.ac.in/faculty/nagamma-patil}{Associate Prof. Nagamma Patil}, NITK, Mangalore, Karnataka / Sept 2019 - March 2020}  
	\begin{itemize}\itemsep \isep                  
    \item Designed and implemented the protein tool hosting using NGINX, Gunicorn, Postgres, systemd services on the Ubuntu server.
	\item Security against DoS/DDoS attacks achieved through NGINX rate limiting and Google CAPTCHA.
	\end{itemize}
\subsection*{Skills}
\begin{itemize}
\item  \textbf{Language \& Frameworks:} Bash (5 years), Python (3 years) - Django, Java (2 years) - Vert.x, C (2 years).
\item \textbf{Technologies:} Git, Linux, Ansible, Cloud (AWS, Azure), Containers, Docker Swarm, K8s,
NGINX.
\item \textbf{Databases:} Elasticsearch, Postgres, Redis, immudb.
\item \textbf{Security and Privacy:} API, Software and Operational Security, PETs - TEEs, SMPC 
\item  \textbf{Project management:} Github projects, Mattermost Open Source Software Development
\item \textbf{Certifications:} \href{https://www.linkedin.com/in/abhi5782-/details/certifications/}{The Kubernetes Crusade: Workshop on Defending and Attacking Kubernetes}, \href{https://elearn.nptel.ac.in/Ecertificate/?q=NPWS22369227122778}{Demystifying JWT, OAuth \& OIDC, NPTEL Workshop}, \href {https://www.youracclaim.com/badges/848a4c99-ab00-449f-afed-59bc88c6a0cd/linked_in_profile}{Google IT Support Professional Certificate}, \href{https://www.credly.com/badges/c2fbeef3-6d43-4213-af28-0f68dedd3334/linked_in_profile}{LFD103: A Beginner's Guide to Linux Kernel Development, Linux Foundation}, \href{https://www.coursera.org/account/accomplishments/certificate/TE3G9PXUEUTR}{Open Source Software Development Methods} 
\item \textbf{Workshop and Presentations:} Smart City Stack and DevOps - \href{https://icdds.org/smartcity.html}{ICDDS, NITK, 2023}, \href{https://edu.ieee.org/in-cucs/events/smart-city-stack-and-devops/}{CHRIST University}
\end{itemize}


\end{document}

